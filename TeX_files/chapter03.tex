\chapter{HOPG Substrate Preparation}
\section{HOPG Substrate and Sample Plate Compatibility}
The holder we use with the threated posts and screws in standard configuration (sample plate, then post and nut to hold post, then holding leaves and another nut), can only accommodate HOPG substrate that is $>1$mm thick. This is because the nuts that go under the screws are 1mm thick so this is the minimum substrate thickness to be able to hold substrate in place. If we use anything thinner, will either need to remove the nuts that are between sample plate and leaves or add a spacer (i.e. another thin piece of HOPG). It seems risky to remove the  nuts since they provide additional support for the posts.

Also the $10\times10$ SPI substrate needs to be cut in half by the machine shop. The HOPG substrates could have a preferred cleaving orientation, so it might be wise to check the cleaving first and specify to machine shop in which direction to cut. We cleave the sample by pulling off tape lengthwise. \textbf{Make sure to let machine shop now that this is a very brittle sample. They have a different method of cutting it.} If they don't know what we gave them, the cut quality may not be so good (personal experience).

\section{Cleaving}
As of May 2017, the following is working procedure for cleaving HOPG:
Use regular Scotch magic tape. I found that the adhesion of the Kapton vacuum tape is poor resulting in an incomplete layer being removed by the tape.
\begin{enumerate}
\item	Put HOPG on a kimwipe and stick a piece of tape to it.
\item	With back of tweezers (tape tweezers to finger), very gently and applying no downward force rub the tape in different directions for 2-3 minutes. The purpose is too remove bubbles and improve adhesion between tape and HOPG.
\item	Peel the tape very slowly (over course of 30-60sec).
\item	After cleaving, do not touch the HOPG cleaved surface with anything. To remove any possible tape residues, sonicate in acetone and IPA for 5 min each.
\item	Let the HOPG dry, then carefully install the substrate in the substrate holder.\\
\end{enumerate}
Note: The cleaving will never be perfect. Will always have non-smooth areas, terraces. However should see sample is relatively flat, no large chunks sticking out, and have some flat terraces. Usually 1 careful cleaving is enough to achieve this.
\section{Assembling sample plate}
This is difficult to describe in words. The best way to do it right is to pay attention to  how you disassemble the existing plate. Briefly, insert each post and screw it in a little, but make sure it doesn't protrude from the other side. Then add a nut to the post and use the special molybdenum tool to tighten it and the post. Then add the leaves, and the nuts on top to hold the leaves down. \textbf{Do not touch the parts or sample with stainless steel tweezers or other tools since we are concerned about Ni contamination}. If can't find a suitable tool, can wrap the stainless tool in foil before using it, but it's not too convenient.  

\textbf{Never sonicate the assembled parts together. It damages the threads}. If parts need cleaning, need to sonicate the individual parts.
\section{Annealing}
Once we load the newly cleaved HOPG to the heater stage we prepare/further clean substrate before deposition by annealing it using resistive heating. Manipulator position is not important for this as long as it is not too close to evaporators (not too low).
\begin{enumerate}
\item Select \emph{resistive heating} in the bias controller touchscreen menu. Turn on resistive heater. Press \emph{prev} and make sure that current is 0. 
\item	Gradually increase current until the base plate temperature (displayed temperature) is 350C. (This corresponds to the substrate temperature of ~450C.) When increase current, need to wait a bit to give the temperature time to warm up. 
\item	Anneal for 30-60min at 350C.
\item	Increase current further so temperature goes to 375C. Anneal for 30min.
\item	If planning on doing deposition the same day, gradually tune current down to the value necessary for deposition. It is safe to let sample be heated for some time (up to hours) at deposition temperature before deposition if you need time to prepare things. Otherwise gradually tune current to zero.
\end{enumerate}